\pdfoutput=1
\documentclass[10pt,twoside]{article}
\usepackage{microtype}
\usepackage{tikz}
\usepackage{amssymb, amsmath, mathrsfs, graphicx} % remove unneeded packages 
\usepackage{fancyhdr, lastpage, titlesec}
\usepackage[colorlinks]{hyperref}
\usepackage[margin=1in]{geometry}
\pagestyle{fancy}
%\setlength{\columnsep}{2em}
\fancyhead[LO,RE]{Math 275, A. Smith}
\fancyhead[RO,LE]{Page \thepage\ of \pageref{LastPage}}
\chead{\textbf{Test 1. Referee Rubric}}
\cfoot{}
%\parindent 0em
\newcommand{\A}[1]{%
   \begin{tikzpicture}
     \node[draw,circle,inner sep=1pt] {#1};
   \end{tikzpicture}}
\newcommand{\E}[1]{%
   \begin{tikzpicture}
     \node[draw,rectangle, inner sep=5pt] {#1};
   \end{tikzpicture}}


\newcommand{\blank}[1]{\underline{\hspace{#1}}}
\renewcommand{\labelenumi}{\textbf{\thesection.\arabic{enumi}.}}
\newcommand{\las}{\href{http://uwstout.courses.wisconsin.edu/d2l/shibbolethSSO/wayf.d2l}{Learn@UW-Stout }}
\newcommand{\ww}{\href{https://webwork.uwstout.edu/webwork2}{WeBWorK }}


\titleformat{\section}[runin]
{\large\bfseries}
{Part \thesection.}{0pt}{}
\titlespacing{\section}
{\parindent}{2em}{1em}



\newcommand{\edit}[1]{{\color{orange} {\tt((}#1{\tt))}}}
\begin{document}

\setcounter{section}{-1}



\section{~Referee Instructions}
To referee, follow this procedure
\begin{enumerate}


\item \textbf{Read this rubic} \emph{first} so you are sure you understand the problem

\item If you find an error in this rubric, \textbf{tell me immediately} so I
can fix it for everyone.

\item Read the author's solution on its own, without referring to the rubric
directly.  Ask yourself ``Did they communicate what they are doing, and why?''
and ``Is this complete, or is there something they forgot to mention?''  You
are a human, not a computer; however, you should \emph{not} have to extrapolate
their intent --- do not think for them!  If you know a solution is correct, but
not for the reasons stated, then make a note of it.  \textbf{Make comments, as
a human, about what is and isn't clear.}  

\item Read each solution a second time.  This time, follow the rubric to assign a \textbf{numerical score to each
problem}.  

\item Write the \textbf{numerical scores and your Referee Code} on the author's
submission.

\item I will personally review all of these comments and scores.  Good commentary and correct grades are worth 4 points to the Referee.

\item Sometimes, people find solutions that are completely unlike what I was expecting.  If you find that the solution does not fit the rubric at all, please make a note of where it stops making sense to you, and do not assign a number.

\end{enumerate}

\newpage
\section{ }
{\color{red}Perform row-reduction to solve this system.  (Hint!  The final solution is $(x,y,z) =
(4,6,8)$.  You still need to do the work, though.)
\[
\begin{cases}
4x + 24y +0z = 160\\
13x + 47y - 6z = 286\\
12x + 78y + 1z = 524\\
\end{cases}\]
}

Convert to augmented matrix and row-reduce.  Here's one way.
\[\begin{pmatrix}
4 & 24 & 0 & 160\\
13 & 47 & -6 & 286\\
12 & 78 & 1 & 524
\end{pmatrix}, \qquad\text{divide row1 by 4}
\]
\[\begin{pmatrix}
1 & 6 & 0 & 40\\
13 & 47 & -6 & 286\\
12 & 78 & 1 & 524
\end{pmatrix} \qquad\text{subtract row3 from row2}\]
\[\begin{pmatrix}
1 & 6 & 0 & 40\\
 1 & -31 & -7 & -238\\
12 & 78 & 1 & 524
\end{pmatrix}\qquad\text{replace row3 with row3-12*row1}\]
\[\begin{pmatrix}
1 & 6 & 0 & 40\\
1 & -31 & -7 & -238\\
0 & 6 & 1 & 44
\end{pmatrix} \qquad\text{subtract row1 from row2}\]
\[\begin{pmatrix}
1 & 6 & 0 & 40\\
0 & -37 & -7 & -278\\
0 & 6 & 1 & 44
\end{pmatrix} \qquad\text{multiply row2 by 6, and row3 by 37}\]
\[\begin{pmatrix}
1 & 6 & 0 & 40\\
0 & -222 & -42 & -1668\\
0 & 222 & 37 & 1628
\end{pmatrix} \qquad\text{add row2 to row3}\]
\[\begin{pmatrix}
1 & 6 & 0 & 40\\
0 & -222 & -42 & -1668\\
0 & 0 & -5 & -40
\end{pmatrix} \qquad\text{add row2 to row3}\]
\[\begin{pmatrix}
1 & 6 & 0 & 40\\
0 & -222 & -42 & -1668\\
0 & 0 & -5 & -40
\end{pmatrix} \qquad\text{divide row3 by $-5$}\]
\[\begin{pmatrix}
1 & 6 & 0 & 40\\
0 & -222 & -42 & -1668\\
0 & 0 & 1 & 8
\end{pmatrix} \qquad\text{add 42 row3 to row2}\]
\[\begin{pmatrix}
1 & 6 & 0 & 40\\
0 & -222 & 0 & -1332\\
0 & 0 & 1 & 8
\end{pmatrix} \qquad\text{divide row2 by -222}\]
\[\begin{pmatrix}
1 & 6 & 0 & 40\\
0 & 1 & 0 & 6\\
0 & 0 & 1 & 8
\end{pmatrix} \qquad\text{subtract 6row2 from row1}\]
\[\begin{pmatrix}
1 & 0 & 0 & 4\\
0 & 1 & 0 & 6\\
0 & 0 & 1 & 8
\end{pmatrix}.\]
Therefore, the solution $(x,y,z) = (4,6,8)$.


\begin{itemize}
\item\textcolor{blue}{This problem is worth 5 points.}
\item\textcolor{blue}{Subtract 2 points they did something other than
row-reduction.}
\item\textcolor{blue}{Subtract 1 point they did not end with 
\[ \begin{pmatrix}1 & 0 & 0 & 4\\0 & 1 & 0 & 6\\ 0 & 0 & 1 & 8\\
\end{pmatrix}\]}
\item\textcolor{blue}{Subtract 1 point for each step of row-reduction where
something went wrong with the arithmetic.  Referee should check if something went wrong by verifying that
$(4,6,8)$ satisfies each of the equations.}
\item \textcolor{blue}{If they caught an arithmetric error, and noted it, but
were unable to fix it, credit back 1 point.  (can only be used once.)}
\end{itemize}


\section{ }
{\color{red}Write down all of the $3 \times 4$ reduced row-echelon form matrices that have
rank 2.\\
(Use $\ast$ to indicate entries that could be any number.)}

Rank 2 means we have to choose the position of 2 pivots.  There are only 4
columns, so there are $\binom{4}{2} = 6$ possible forms.
\[
\begin{pmatrix}
1 & 0 & * & *\\
0 & 1 & * & *\\
0 & 0 & 0 & 0
\end{pmatrix},\qquad
\begin{pmatrix}
1 & * & 0 & *\\
0 & 0 & 1 & *\\
0 & 0 & 0 & 0
\end{pmatrix},\qquad
\begin{pmatrix}
1 & * & * & 0\\
0 & 0 & 0 & 1\\
0 & 0 & 0 & 0
\end{pmatrix}\]
\[
\begin{pmatrix}
0 & 1 & 0 & *\\
0 & 0 & 1 & *\\
0 & 0 & 0 & 0
\end{pmatrix},\qquad
\begin{pmatrix}
0 & 1 & * & 0\\
0 & 0 & 0 & 1\\
0 & 0 & 0 & 0
\end{pmatrix},\qquad
\begin{pmatrix}
0 & 0 & 1 & 0\\
0 & 0 & 0 & 1\\
0 & 0 & 0 & 0
\end{pmatrix}\]

\begin{itemize}
\item\textcolor{blue}{This problem is worth 2 points.}
\item\textcolor{blue}{Subtract 0.5 point for each matrix that is not in
RREF.}
\item\textcolor{blue}{Subtract 0.25 point for each matrix that is missing
$\ast$'s where they should be.}
\item\textcolor{blue}{Subtract 0.5 point for each missing case.}
\end{itemize}


\section{ }
{\color{red}Suppose that $A$ is a $4 \times 4$ matrix for which the following procedure
yields $I$.  First, swap $\rho_2$ and $\rho_4$.  Second, replace
$\rho_1$ with $\rho_1 + 5\rho_3$.  Third, multiply $\rho_3$ by 6, and divide
$\rho_1$ by 5.  Finally, replace $\rho_4$ with $\rho_4 - 3 \rho_1$.
Recall that these row-operations can be un-done in the correct order, to change $I$ back to $A$.
Find the matrix $A$.
}


The first row-operation is 
\[
E_1 = \begin{pmatrix}
1 & 0 & 0 & 0\\
0 & 0 & 0 & 1\\
0 & 0 & 1 & 0\\
0 & 1 & 0 & 0
\end{pmatrix}\]
The second row-operation is 
\[
E_2 = \begin{pmatrix}
1 & 0 & 5 & 0\\
0 & 1 & 0 & 0\\
0 & 0 & 1 & 0\\
0 & 0 & 0 & 1
\end{pmatrix}\]
The third row-operation is 
\[
E_3 = \begin{pmatrix}
\frac15 & 0 & 0 & 0\\
0 & 1 & 0 & 0\\
0 & 0 & 6 & 0\\
0 & 0 & 0 & 1
\end{pmatrix}\]
The fourth row-operation is 
\[
E_4 = \begin{pmatrix}
1 & 0 & 0 & 0\\
0 & 1 & 0 & 0\\
0 & 0 & 1 & 0\\
-3 & 0 & 0 & 1
\end{pmatrix}\]
And, we have 
\[ E_4E_3E_2E_1 A = I\]
Therefore,
\[ A = E_1^{-1}E_2^{-1}E_3^{-1}E_4^{-1}I \]

Hence, undoing each of the above operations in the correct order, we have
\[
A =
\begin{pmatrix}
1 & 0 & 0 & 0\\
0 & 0 & 0 & 1\\
0 & 0 & 1 & 0\\
0 & 1 & 0 & 0
\end{pmatrix}
\begin{pmatrix}
1 & 0 & -5 & 0\\
0 & 1 & 0 & 0\\
0 & 0 & 1 & 0\\
0 & 0 & 0 & 1
\end{pmatrix}
\begin{pmatrix}
5 & 0 & 0 & 0\\
0 & 1 & 0 & 0\\
0 & 0 & \frac16 & 0\\
0 & 0 & 0 & 1
\end{pmatrix}
\begin{pmatrix}
1 & 0 & 0 & 0\\
0 & 1 & 0 & 0\\
0 & 0 & 1 & 0\\
+3 & 0 & 0 & 1
\end{pmatrix}
=
\begin{pmatrix}
5 & 0 & -\frac56 & 0\\
3 & 0 & 0 & 1\\
0 & 0 & \frac16 & 0\\
0 & 1 & 0 & 0
\end{pmatrix}
\]

\begin{itemize}
\item\textcolor{blue}{This problem is worth 3 points.}
\item\textcolor{blue}{Subtract 1 point for not converting the row-operations to
elementary matrices.}
\item\textcolor{blue}{Subtract 0.5 points for each incorrect elementary
matrix.}
\item\textcolor{blue}{Subtract 1 point for not un-doing each row-operations to
find its respective inverse.  (Note!  It is fair to jump straight to this,
without writing down the original row-ops).}
\item\textcolor{blue}{Subtract 0.5 points for each incorrect inverse.}
\item\textcolor{blue}{Subtract 1 point for not multiplying the inverses in the
correct order}
\item\textcolor{blue}{Subtract 0.5 points (total) for arithmetic errors in
matrix multiplication.}
\end{itemize}

\section{ }
{\color{red}Suppose that $A$ is a $3 \times 3$ matrix and that $E_1, E_2, \ldots, E_k$ are
elementary matrices such that 
\[ 
 E_k \cdots E_2 E_1 A =  R = \begin{pmatrix}
1 & 0 & \frac15\\
0 & 1 & -3\\
0 & 0 & 0
\end{pmatrix},\quad\text{and}\quad
E_k \cdots E_2 E_1 
=
 \begin{pmatrix}
4 & 2 & 8\\
8 & 1 & 3\\
0 & 1 & 1
\end{pmatrix}.
\]}
\begin{enumerate}
\item
{\color{red}Find the solution of the system $A\vec{x}=\vec{0}$ in parametric form.}

The solution to $A\vec{x}=\vec{0}$ is the same as the solution to $R \vec{x} =
\vec{0}$, since row operations don't change the 0s.  Therefore, using $x_3$ as
a free variable, the solution is
\[ 
\begin{pmatrix}
x_1\\x_2\\x_3
\end{pmatrix}
= \begin{pmatrix}
- \frac15 \\ 3 \\ 1\end{pmatrix}t
\]
This is a line through the origin.  By rescaling $t$, it could be written
without fractions as
\[ 
\begin{pmatrix}
x_1\\x_2\\x_3
\end{pmatrix}
= \begin{pmatrix}
- 1 \\ 15 \\ 5\end{pmatrix}t
\]


\begin{itemize}
\item\textcolor{blue}{This sub-problem is worth 2 points.}
\item\textcolor{blue}{Subtract 1 point for not writing $Rx=0$ or equivalent, or
for not recognizing that $Ax=0$ and $Rx=0$ have the same solution.}
\item\textcolor{blue}{Subtract 1 point for completely incorrect parametric
form.}
\item\textcolor{blue}{Subtract 0.5 points for $\pm$ mistakes in the
parametric form.}
\end{itemize}


\item 
{\color{red}Write the solution to the inhomogeneous system 
\[ A \begin{pmatrix}
x_1 \\ x_2 \\ x_3
\end{pmatrix} = \begin{pmatrix}\frac12 \\ -3 \\ 3\end{pmatrix},\]
or explain precisely why no solution exists.
}

The solution to $A\vec{x}=\vec{b}$ is the same as the solution to $R \vec{x} =
\vec{c}$ where $\vec{c} = 
E_k\cdots E_2E_1 \vec{b}$.  That vector is
\[ 
\vec{c} = E_k \cdots E_2 E_1\vec{b} 
=
 \begin{pmatrix}
4 & 2 & 8\\
8 & 1 & 3\\
0 & 1 & 1
\end{pmatrix}
\begin{pmatrix}\frac12 \\ -3 \\ 3\end{pmatrix}
= 
\begin{pmatrix}
20 \\ 10 \\ 0
\end{pmatrix}\]
So, we are solving the system
\[
\begin{cases}
x_1 + 0x_2 + \frac15 x_3 = 20\\
x_2 -3x_3 = 10\\
0 = 0
\end{cases}\]
The solution is 
\[ 
\begin{pmatrix}
x_1\\x_2\\x_3
\end{pmatrix}
= 
\begin{pmatrix}20\\10\\0\end{pmatrix}+
\begin{pmatrix}
- \frac15 \\ 3 \\ 1\end{pmatrix}t
\]

\begin{itemize}
\item\textcolor{blue}{This sub-problem is worth 2 points.}
\item\textcolor{blue}{Subtract 1 point for not finding $c = (20,10,0)$ at
all.}
\item\textcolor{blue}{Subtract 0.5 point for arithmetic mistake in finding
$c$.}
\item\textcolor{blue}{Subtract 1 point for not writing $Rx=c$ or equivalent, or
for not recognizing that $Ax=b$ and $Rx=c$ have the same solution.}
\item\textcolor{blue}{Subtract 1 point for completely incorrect parametric
form.}
\item\textcolor{blue}{Subtract 0.5 points for $\pm$ arithmetic mistakes in the
parametric form.}
\end{itemize}


\item 
{\color{red}Write the solution to the inhomogeneous system 
\[ A \begin{pmatrix}
x_1 \\ x_2 \\ x_3
\end{pmatrix} =  \begin{pmatrix}1 \\ 39 \\ -9\end{pmatrix},\]
or explain precisely why no solution exists.
}
\end{enumerate}

The solution to $A\vec{x}=\vec{b}$ is the same as the solution to $R \vec{x} =
\vec{c}$ where $\vec{c} = 
E_k\cdots E_2E_1 \vec{b}$.  That vector is
\[ 
\vec{c} = E_k \cdots E_2 E_1\vec{b} 
=
 \begin{pmatrix}
4 & 2 & 8\\
8 & 1 & 3\\
0 & 1 & 1
\end{pmatrix}
\begin{pmatrix}1 \\ 39 \\ -9\end{pmatrix}
=
\begin{pmatrix}
10 \\ 20 \\ 30 
\end{pmatrix}\]

Note that $R\vec{x} = \vec{c}$ has the row $0 = 30$.  The system in
inconsistent -- there is not solution.

\begin{itemize}
\item\textcolor{blue}{This sub-problem is worth 1 point.}
\item\textcolor{blue}{Subtract 0.5 point for not finding $c = (10,20,30)$ at
all.}
\item\textcolor{blue}{Subtract 0.25 point for arithmetic mistake in finding
$c$.}
\item\textcolor{blue}{Subtract 0.75 point for not recognizing the inconsistent
system.}
\end{itemize}


\section{ }
{\color{red}The following pictures show two lines, corresponding to a system of two linear equations
in two variables, $[A|b]$.
For each picture, answer these questions:}
\begin{itemize}
\item {\color{red}Is the system homogeneous or inhomogeneous?}
\item {\color{red}How many solutions does the system have?}
\item {\color{red}What is the rank of the $2 \times 2$ matrix $A$?}
\item {\color{red}What is the rank of the $2 \times 3$ augmented matrix $[A|b]$?}
\end{itemize} 

(removed images for brevity)

The first image is two identical lines through the origin.
This would arise from a system of the form 
\[\begin{pmatrix} 1 & * &|0 \\ 0 & 0 &|0 \end{pmatrix}\]
This is homogeneous (the solution contains $(0,0)$), and the solution is a line
(1-dimensional, with infinitely many solutions).  $A$ has rank 1, and $[A|b]$
also has rank 1, as seen above.

The second image is parallel identical lines. 
This would arise from a system of the form 
\[\begin{pmatrix} 1 & * &|0 \\ 0 & 0 & |1 \end{pmatrix}\]
This is inhomogeneous (the solution does not contain $(0,0)$), and the solution
is a empty.  There are no solutions.
$A$ has rank 1, and $[A|b]$
has rank 2, as seen above.

The third image is two identical lines that miss the origin.
This would arise from a system of the form 
\[\begin{pmatrix} 1 & * &|* \\ 0 & 0 & |0 \end{pmatrix}\]
This is inhomogeneous (the solution does not contain $(0,0)$), and the solution
is a line (1-dimensional, with infinitely many solutions). 
$A$ has rank 1, and $[A|b]$
also has rank 1, as seen above.

The fourth image is two lines that cross at a point that is not the origin.
This would arise from a system of the form 
\[\begin{pmatrix} 1 & 0 &|* \\ 0 & 1 & |* \end{pmatrix}\]
This is inhomogeneous (the solution does not contain $(0,0)$), and the solution
is a single point (0-dimensional, with exactly one solution).
$A$ has rank 2, and $[A|b]$
also has rank 2, as seen above.


\begin{itemize}
\item\textcolor{blue}{This sub-problem is worth 5 points.}
\item\textcolor{blue}{Subtract 0.5 point for each incorrect
homogeneous/inhomogenous}
\item\textcolor{blue}{Subtract 0.5 point each incorrect statement about the
number of solutions.  In this problem, it is acceptable to say
``1-dimensional'' or ``infinitely many'' interchangeably.  Also, one may say
``0-dimensional'' or ``exactly one'' interchangeably.}
\item\textcolor{blue}{Subtract 0.5 point each incorrect rank statement.}
\item\textcolor{blue}{Add back 1 point if someone mis-stated the ranks, but
write an example system or wrote a parametrization of the solution correctly.
This can be used only once.}
\end{itemize}




%%%% END %%%%
\end{document}

